\documentclass[12pt,a4paper]{article}

% --- Packages ---
\usepackage[utf8]{inputenc}
\usepackage{graphicx}
\usepackage{hyperref}
\usepackage{listings}
\usepackage{xcolor}
\usepackage{geometry}
\geometry{margin=1in}
\usepackage{titlesec}
\usepackage{enumitem}
\usepackage{fancyhdr}

% --- Header / Footer ---
\pagestyle{fancy}
\fancyhf{}
\lhead{AutoCull User Documentation}
\rhead{\thepage}

% --- Hyperlink setup ---
\hypersetup{
    colorlinks=true,
    linkcolor=blue,
    urlcolor=cyan,
    pdftitle={AutoCull End User Documentation},
    pdfauthor={Kevin Wang},
    pdfsubject={Image Management Software},
    pdfkeywords={AutoCull, photo sorting, duplicate detection, user guide}
}

% --- Code block style ---
\lstdefinestyle{console}{
    backgroundcolor=\color{gray!10},
    basicstyle=\ttfamily\footnotesize,
    frame=single,
    breaklines=true
}

% --- Section styling ---
\titleformat{\section}{\Large\bfseries}{\thesection}{1em}{}
\titleformat{\subsection}{\large\bfseries}{\thesubsection}{1em}{}

% --- Document Info ---
\title{\textbf{AutoCull End-User Documentation}}
\author{Group 11}
\date{\today}

\begin{document}
\maketitle
\tableofcontents
\newpage

% -------------------------------------------------------------------------
\section{Introduction}

\textbf{AutoCull} is an intelligent image management application that automatically identifies and removes near-duplicate or low-quality photos from a given collection. The program leverages AI/ML techniques to both streamline large-scale photo curation, and improve the user's photographic abilities.

This document serves as a comprehensive guide for end users, covering installation, usage, configuration, and troubleshooting.

% -------------------------------------------------------------------------
\section{System Requirements}

\begin{itemize}
    \item Operating System: Windows 10 or later / macOS / Linux
    \item Python 3.10 or higher
    \item PostgreSQL 14 or higher
    \item Recommended RAM: 8 GB+
    \item Disk space: Minimum 1 GB free
\end{itemize}


% -------------------------------------------------------------------------
\section{Installation Guide}

\subsection{Step 1: Obtain AutoCull}

You can install AutoCull using one of the following two methods:

\subsubsection*{Option A: Download the Pre-Built Executable (Recommended for Most Users)}
\begin{enumerate}[label=\arabic*.]
    \item Visit the project's GitHub repository:  
    \url{https://github.com/Manwithacape/COSC345---Group_11/actions}
    \item Under the latest successful build, scroll to the \textbf{Artifacts} section.
    \item Download the file named \texttt{AutoCull.exe}.
    \item Follow steps below (3.4 Step 4) to set up the PostgreSQL database.
    \item Once downloaded, simply double-click \texttt{AutoCull.exe} to launch the application.  
    No setup or Python installation is required.
\end{enumerate}

\subsubsection*{Option B: Clone and Run the Source Code (For Developers)}
\begin{lstlisting}[style=console]
git clone https://github.com/Manwithacape/COSC345---Group_11.git
cd COSC345---Group_11
\end{lstlisting}

\subsection{Step 2: Create a Virtual Environment (Source Installation Only)}
\begin{lstlisting}[style=console]
python -m venv venv
source venv/bin/activate       # macOS/Linux
venv\Scripts\activate          # Windows
\end{lstlisting}

\subsection{Step 3: Install Dependencies (Source Installation Only)}
\begin{lstlisting}[style=console]
pip install -r requirements.txt
\end{lstlisting}

\subsection{Step 4: Set Up PostgreSQL Database}

AutoCull uses PostgreSQL to store image metadata, similarity metrics, and user preferences.

\begin{enumerate}[label=\arabic*.]
    \item Install PostgreSQL 14 or higher from \url{https://www.postgresql.org/download/}.
    \item During installation, note your chosen \textbf{username}, \textbf{password}, and \textbf{port number} (default is 5432).
    \item Open the PostgreSQL interactive terminal:
    \begin{lstlisting}[style=console]
    psql -U postgres
    \end{lstlisting}
    \item Create a new database and user for AutoCull:
    \begin{lstlisting}[style=console]
    CREATE DATABASE autocull_db;
    CREATE USER autocull_user WITH PASSWORD 'yourpassword';
    GRANT ALL PRIVILEGES ON DATABASE autocull_db TO autocull_user;
    \end{lstlisting}

    Example configuration file (\texttt{.env} or \texttt{config.json}):
    \begin{lstlisting}[style=console]
    {
        "dbname": "autocull_db",
        "user": "autocull_user",
        "password": "yourpassword",
        "host": "localhost",
        "port": 5432
    }
    \end{lstlisting}
    \item Initialize the schema using the provided \texttt{schema.db} file.  
    You should see a confirmation message indicating that tables were created successfully.
\end{enumerate}

\subsection{Step 5: Launch AutoCull}

\begin{itemize}
    \item \textbf{If using the .exe file:}  
    Simply double-click \texttt{AutoCull.exe} to start the program.
    \item \textbf{If running from source:}
    \begin{lstlisting}[style=console]
    python app.py
    \end{lstlisting}
\end{itemize}

If the program launches successfully, you should see the AutoCull interface.


% -------------------------------------------------------------------------

\section{Configuration}

AutoCull requires a database configuration file named \texttt{db\_config.json} or environment variables defined in a \texttt{.env} file in the project root.

\begin{lstlisting}[style=console]
{
    "dbname": "autocull_db",
    "user": "autocull_user",
    "password": "yourpassword",
    "host": "localhost",
    "port": 5432
}
\end{lstlisting}

Ensure this file is present before running the application.


% -------------------------------------------------------------------------
\section{Dependencies}

AutoCull relies on several Python libraries for image processing, database interaction, and machine learning.  
All dependencies can be installed automatically using the \texttt{requirements.txt} file included with the project.

\begin{itemize}
    \item \textbf{ImageHash} – Perceptual hashing for duplicate detection
    \item \textbf{numpy} – Numerical operations and data manipulation
    \item \textbf{opencv\_python} – Image analysis and computer vision tools
    \item \textbf{piexif} – EXIF metadata reading and writing
    \item \textbf{Pillow} – Image loading, display, and format conversion
    \item \textbf{python-dotenv} – Environment variable and configuration management
    \item \textbf{scikit\_learn} – Machine learning and clustering algorithms
    \item \textbf{scikit-image} – Advanced image processing utilities
    \item \textbf{ttkbootstrap} – Modern themed interface for Tkinter
    \item \textbf{rawpy} – RAW image file decoding
    \item \textbf{psycopg2-binary} – PostgreSQL database connectivity
    \item \textbf{google-genai} – Integration for Google Gemini-based text generation
    \item \textbf{torch} – Core PyTorch deep learning framework
    \item \textbf{torchvision} – Vision utilities for PyTorch models
    \item \textbf{transformers} – Access to pretrained deep learning models
    \item \textbf{face\_recognition} – Face detection and recognition functionality
    \item \textbf{exifread} – Lightweight EXIF metadata extraction
    \item \textbf{CLIP (OpenAI)} – Image-text embedding model for similarity and semantic scoring
\end{itemize}

To install all dependencies:
\begin{lstlisting}[style=console]
pip install -r requirements.txt
\end{lstlisting}


% -------------------------------------------------------------------------

\section{Key Features}

\begin{itemize}
    \item Intelligent duplicate detection using perceptual hashing
    \item Photo scoring based on exposure, focus, and composition
    \item EXIF metadata viewer
    \item Collection management interface
    \item Integrated PostgreSQL backend for image data
    \item Lightweight, dark-themed Tkinter GUI
\end{itemize}


% -------------------------------------------------------------------------
\section{Using AutoCull}

\subsection{Basic Workflow}

\begin{enumerate}[label=\arabic*.]
    \item Launch AutoCull.
    \item Import images by selecting "Import" from the sidebar.
    \item AutoCull scans for near duplicates and low-quality images.
    \item Review detected clusters of similar images.
    \item Suggest photos for deletion based on quality scores.
    \item Confirm deletion or move duplicates to a separate folder.
\end{enumerate}

\subsection{GUI}
The GUI provides options for:
\begin{itemize}
    \item Photo importing
    \item Viewing detected duplicates
    \item Marking photos for deletion
    \item Previewing detected duplicates
    \item Batch delete/move operations
    \item Suggesting photos for deletion based on quality scores
    \item Viewing collections
    \item Creating collections
    \item Viewing exif data
    \item Viewing image metrics
\end{itemize}

% -------------------------------------------------------------------------
\section{Troubleshooting}

\begin{itemize}
    \item \textbf{Program doesn’t start:}  
      Ensure Python 3.10+ is installed and dependencies are correctly installed.

    \item \textbf{Database connection error:}  
      Verify that PostgreSQL is running and credentials are correct.

    \item \textbf{No duplicates found:}  
      Try lowering the \texttt{threshold} value.

    \item \textbf{Permission errors:}  
      Run as administrator or check write permissions on output directories.
\end{itemize}

% -------------------------------------------------------------------------

\section{Uninstallation}

To remove AutoCull, delete the application folder and its virtual environment.  
If desired, remove the PostgreSQL database using:

\begin{lstlisting}[style=console]
DROP DATABASE autocull_db;
DROP USER autocull_user;
\end{lstlisting}


% -------------------------------------------------------------------------
\section{About}

This application was developed as part of a university project for COSC345. For more information, visit the GitHub repository: \url{https://github.com/Manwithacape/COSC345---Group_11}

Please do not attempt to contact anyone regarding bugs, issues, or feature requests. These will likely not be addressed.

% -------------------------------------------------------------------------
\section{Credits and License}

Developed by \textbf{Group 11, COSC345 2025 S2} as part of the AutoCull project.  

Licensed under the MIT License.

THE SOFTWARE IS PROVIDED "AS IS", WITHOUT WARRANTY OF ANY KIND, EXPRESS OR IMPLIED, INCLUDING BUT NOT LIMITED TO THE WARRANTIES OF MERCHANTABILITY, FITNESS FOR A PARTICULAR PURPOSE AND NON-INFRINGEMENT. IN NO EVENT SHALL THE AUTHORS OR COPYRIGHT HOLDERS BE LIABLE FOR ANY CLAIM, DAMAGES OR OTHER LIABILITY, WHETHER IN AN ACTION OF CONTRACT, TORT OR OTHERWISE, ARISING FROM, OUT OF OR IN CONNECTION WITH THE SOFTWARE OR THE USE OR OTHER DEALINGS IN THE SOFTWARE.


\end{document}
